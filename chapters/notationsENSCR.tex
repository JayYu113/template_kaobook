{\usekomafont{chapter}Notations} \\[2ex]
\addcontentsline{toc}{chapter}{Notations} % Add the preface to the table of contents as a chapter

Notations mathématiques utilisées dans ce cours.
\begin{center}
    \begin{tabular}{@{} ll @{}} % Column formatting, @{} suppresses leading/trailing space
       \toprule	  
       \textbf{Notation}			& \textbf{Signification}\\
       \midrule
	   $\stackrel{\text{def}}=$					&relation de définition\\
	   $\sim$						&égal en ordre de grandeur\\
       \cmidrule(r){1-2}
	   $[x]$						&dimension de la grandeur $x$\\
	   $\sigma_x$					&incertitude-type de la grandeur $x$\\
	   $\Delta x$					&incertitude élargie (à 95\%) de la grandeur $x$\\
	   $\overline{x}$				&moyenne (espérance) d'une grandeur aléatoire $x$\\
	   $\mathrm{d} x/\mathrm{d} t$	ou $\dot x$&dérivée première par rapport au temps\\
	   $\mathrm{d}^n x/\mathrm{d} t^n$ ou $\ddot x$	&dérivée n-ième par rapport au temps\\
       \cmidrule(r){1-2}
	   $\overrightarrow{u}$			&vecteur unitaire\\
	   $(\overrightarrow{u_x},\overrightarrow{u_y},\overrightarrow{u_z})$	&base cartésienne\\
	   $\|\overrightarrow{A}\|$ ou $A$	&norme du vecteur $\overrightarrow{A}$\\
	   $A_{z}$						&composante suivant l'axe (O$z) =A_{z}=\overrightarrow{A}\cdot\overrightarrow{u_{z}}$\\	
      \cmidrule(r){1-2}
   		$\arcdecercle{AB}$			&longueur de l'arc de cercle AB\\
	    $\log$			&logarithme à base 10\\
	    $\ln$			&logarithme népérien (à base e)\\
 	  \bottomrule
    \end{tabular}
\end{center}

